\documentclass{article}

% Language setting
% Replace `english' with e.g. `spanish' to change the document language
\usepackage[english]{babel}

% Set page size and margins
% Replace `letterpaper' with `a4paper' for UK/EU standard size
\usepackage[letterpaper,top=2cm,bottom=2cm,left=3cm,right=3cm,marginparwidth=1.75cm]{geometry}

% Useful packages
\usepackage{amsmath}
\usepackage{graphicx}
\usepackage[colorlinks=true, allcolors=blue]{hyperref}

\title{Combining two results for better sensitivity to new Higgs: HIG-22-007 Physics Briefing}
\author{Stephanie Kwan (skwan@princeton.edu)}

\begin{document}
\maketitle

% \begin{abstract}
% Your abstract.
% \end{abstract}

\section{Is there just one Higgs boson? (157 words)}

The Standard Model theory of particle physics summarizes all known particles and their interactions, but it is not a complete picture of the universe as it cannot account for multiple experimental observations. 

For example, the existence of neutrino mass and dark matter are not addressed by the Standard Model. 

Several beyond the Standard Model theories address these observations while identifying the Higgs boson (discovered in 2012 by the ATLAS and CMS experiments with a mass of 125 GeV) as one of an extended group of Higgs particles. 

How would we go about searching for these new Higgs particles? One scenario well-supported by these theories, is where the 125 GeV Higgs boson decays to two new intermediate Higgs,
which both decay to known Standard Model particles. We observe and measure these known particles in the detector. If we observe a higher number of these known particles than we expect, then we may have evidence of these new particles. 


\section{What did each analysis do, and how do we combine two analyses? (412 words)}

One analysis group studied the case where one intermediate new Higgs decayed into two bottom quarks, and the other into two tau leptons, written h $\rightarrow$ aa $\rightarrow$ 2b 2tau.
The second analysis group considered the case where one intermediate Higgs decayed into two bottom quarks, and the other into two muons, i.e. h $\rightarrow$ aa $\rightarrow$ 2b 2 muons. 

The analyses differ in the muons and taus in their final states, and these particles are extremely different: muons are stable in the detector and are fairly well-reconstructed. 
In comparison, the tau lepton is unstable and decays inside the detector. About 35\% of the time, a tau lepton decays into the lighter leptons (i.e. electrons and muons), and the remaining 65\% of the time, it decays into hadrons, 
which further decay. These possibilities complicate tau reconstruction and distinguishing them from background noise. 

Bottom quarks are heavy enough to decay into jets (showers of particles in a cone), which are also challenging to reconstruct, and difficult to distinguish from low-energy interactions which we are not interested in.

In order to optimize sensitivity to new physics, the two groups independently developed very different strategies.

For instance, the h $\rightarrow$ aa $\rightarrow$ 2b 2tau group trained a machine learning algorithm to identify signal-like events and reject background events. When applied to real data, the algorithm assigns a score from 0 to 1, 
based on how signal-like the data event is. We use this score to categorize real data based into signal regions (where we expect to see the most signal) and control regions (which we expect to see only background). 

In contrast, the h $\rightarrow$ aa $\rightarrow$2b 2mu group distinguished signal-like events from background events by defining a physical variable based on the masses of the two b quarks and two muons, which tends to be very small for signal-like events,
and large for background events. The advantage of this approach is to utilize the excellent muon mass reconstruction provided by the CMS detector. 

Despite these two different approaches, both analyses are looking for an excess of events as a function of the total mass of the pseudoscalars. 

This allows us to combine our results, improving the experiment's overall sensitivity to the exotic Higgs decay, compared to if the two groups' results were only considered individually. 

\section{How do we interpret our results?}



\section{What does this mean for the big picture?}



\end{document}