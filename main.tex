\documentclass{article}

% Language setting
% Replace `english' with e.g. `spanish' to change the document language
\usepackage[english]{babel}

% Set page size and margins
% Replace `letterpaper' with `a4paper' for UK/EU standard size
\usepackage[letterpaper,top=2cm,bottom=2cm,left=3cm,right=3cm,marginparwidth=1.75cm]{geometry}

% Useful packages
\usepackage{amsmath}
\usepackage{graphicx}
\graphicspath{ {./figures/} }
\usepackage[colorlinks=true, allcolors=blue]{hyperref}

\title{Combining two results for better sensitivity to new Higgs: HIG-22-007 Physics Briefing}
\author{Briefing prepared by: \\ Anagha Aravind (aaravind@wisc.edu) \\ Pallabi Das (pdas@princeton.edu) \\ Pieter Everaerts (pieter.everaerts@cern.ch) \\ Elham Khazaie (elham.khazaie@cern.ch) \\ Stephanie Kwan (skwan@princeton.edu) \\ Ho-Fung Tsoi (htsoi@wisc.edu)}

\begin{document}
\maketitle

% \begin{abstract}
% Your abstract.
% \end{abstract}
Total: currently 1094 words (limit is 1000 words)

\section{Is there just one Higgs boson? (187 words)}

The Standard Model theory of particle physics summarizes all known particles and their interactions, but it is not a complete picture of the universe as it cannot account for multiple experimental observations. 

For example, the existence of neutrino mass and dark matter are not addressed by the Standard Model. 

Several beyond the Standard Model theories, such as 2HDM+S (two Higgs doublet models extended by a scalar), address these observations by allowing for an extended group of Higgs particles. The Higgs boson with mass 125 GeV, 
discovered in 2012 by the ATLAS and CMS experiments, would be one of the Higgs particles in this extended group. 

How would we go about searching for these predicted Higgs particles? In one scenario, the 125 GeV Higgs boson can decay to two new intermediate Higgs particles (called ``a" here),
which both decay to known Standard Model particles. If we observe a statistically significant difference between our expected and observed number of events in our ``signal region", i.e. where the measured particles have 
energies and kinematic properties similar to our signal, then we may have evidence of these new particles. 


\section{What did each analysis do, and how do we combine two analyses? (537 words)}

One analysis group studied the case where one ``a" particle decays into two bottom quarks, and the other ``a" decays into two tau leptons, written h $\rightarrow aa \rightarrow 2\tau 2b$.
The second analysis group considered the case where one ``a" decays into two bottom quarks, and the other ``a" decays into two muons, i.e. h $\rightarrow aa \rightarrow 2\mu 2b$. 


These particles have very different signatures in the CMS detector. Muons are stable in the detector (i.e. they do not decay). On the other hand, tau leptons are unstable: about 35\% of the time, a tau will decay into electrons and muons, and the remaining 65\% of the time, it will decays into hadrons, 
which further decay. These possibilities complicate tau reconstruction, and necessitate techniques to distinguish them from jets, which are showers of particles originating from quarks and gluons.
* Bottom quarks decay into particles that tend to travel a small, but measurable distance before producing a jet. This so-called displaced vertex makes it possible to distinguish bottom quarks from other quarks.

The figure below visualizes a real data event that contains two bottom-quark jets and two tau leptons close together in the detector, which is consistent with the h $\rightarrow aa \rightarrow 2\tau 2b$ signature of interest.

\begin{figure}[ht]
    \centering
    \includegraphics[width=8cm]{fireworks_event1/event1_barrel_slice.png}
    \caption{A data event from 2018 which has a similar signature as the theorized decay of a 125 GeV Higgs boson to two intermediate ``a'' particles, where one decays to two bottom-quark jets,
    and the other decays to two tau leptons. Jets are colliminated particles and often appear as a cluster of energy deposits in the hadronic calorimeter (HCAL, shown in blue) and the electromagnetic calorimeter (ECAL, shown in red). 
    As tau leptons are unstable in the detector, in this event, one tau lepton decays into a muon (muon track in red, connecting to the muon energy deposits shown in red squares).
    The other tau lepton decays into hadrons and produces a jet.}
\end{figure}


In order to optimize sensitivity to new physics, the two groups independently developed very different strategies.

For instance, the h $\rightarrow aa \rightarrow 2\tau 2b$ group trained a machine learning algorithm to identify signal-like events and reject background events. The algorithm analyzes real data events and assigns a probability (from low probability to high probability) that a given data event is similar to the signal.
We use this probability to separate our events into signal regions (where we expect to see the most signal) and control regions (which we expect to see only background). Our measurement is performed in the signal region, while we use the control regions to verify that we understand and correctly model
our background events.

In contrast, the h $\rightarrow aa \rightarrow 2\mu 2b$ group distinguished signal-like events from background events by defining a physical variable based on the masses of the two b quarks and two muons, which tends to be very small for signal-like events,
and large for background events. The advantage of this approach is to utilize the excellent muon mass reconstruction provided by the CMS detector. 

Despite these two different approaches, both analyses are looking for an excess of events as a function of the total mass of the pseudoscalars. 

This allows us to combine our results, improving the experiment's overall sensitivity to the exotic Higgs decay, compared to if the two groups' results were only considered individually. 

\section{How do we interpret our results? (287 words)}

If our measurements agree well with predictions of the Standard Model within uncertainties, we can ``rule out", or exclude, hypotheses of new physics. We do this by setting exclusion limits.

The exclusion limits for one of the theory types studied (namely, 2HDM+S type 1) are summarized in the plot below, for each individual analysis and the two analyses combined: 


    \begin{figure}[ht]
        \centering
        \includegraphics[width=8cm]{full_run2_plot_BRaa_Type1.pdf}
        \caption{Expected and observed limits for 2HDM+S Type I models, for h $\rightarrow aa \rightarrow 2\mu 2b$ and h $\rightarrow aa \rightarrow 2\tau 2b$ separately, then combined (h $\rightarrow bb \rightarrow \ell\ell$).}
    \end{figure}

The vertical axis is the probability that the 125 GeV Higgs particle decays into two ``a" particles, termed the ``branching ratio'' (B). This cannot be greater than 100\% (the dashed horizontal line). 

The horizontal axis is the mass of the ``a" particle in this theory type, ranging from 10 GeV to 62.5 GeV. 

Since we cannot measure the ``true" decay probability (branching ratio) of H $\rightarrow aa$, instead we set a lower bound, which should be less than the ``true" value 95\% of the time if we were to repeat the measurement over and over again.

This lower bound is calculated for each possible mass of the ``a" particle in each theory, first without fitting to the signal region of the data (giving the ``expected" limits), then with the signal region included (the ``observed" limit). 

If our observed limit is greater than the expected limit with statistical uncertainty (i.e. the solid shaded area falls above the bold line with dashed shaded area), this would be a sign of
disagreement with the Standard Model. However, since our observed limits fall within the expected limits (including uncertainties), our observations are consistent with the Standard Model. 


\section{What does this mean for the big picture? (83 words)}

By using the full Run 2 dataset from the years 2016 through 2018, and a host of new and improved optimizations, the two analyses were able to set stronger limits than the analyses which used only data from 2016.

Furthermore, combining two measurements provides stronger limits on new hypotheses compared to considering each measurement alone.

These latest results complement and strengthen efforts at the CMS experiment to use the 125 GeV Higgs boson as a probe for searching for signals of new physics. 



\section{References and links}

\begin{itemize}
    \item Exclusion limit figure from \begin{verbatim}https://pdas.web.cern.ch/pdas/HAA/llbb/full_run2_plot_BRaa_Type1.pdf\end{verbatim}
\end{itemize}
\end{document}